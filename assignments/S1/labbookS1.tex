%%%%%%%%%%%%%%%%%%%%%%%%%%%%%%%%%%%%%%%%%%%%%%%%%%%%%%%%%%%%%%%%%%%%%%%%%%%%%%%%%%%%
% Do not alter this block (unless you're familiar with LaTeX)
\documentclass{../labbook}

%%%%%%%%%%%%%%%%%%%%%%%%%%%%%%%%%%%%%%%%%%%%%
%Fill in the appropriate information below
\lhead{Cantao Su}
\rhead{Speech Synthesis I} 
\chead{\textbf{Lab Book S1. Due: \textbf{Fri 01.12.2023 23:59} CET}}
%%%%%%%%%%%%%%%%%%%%%%%%%%%%%%%%%%%%%%%%%%%%%

\begin{document}
\begin{mdframed}[backgroundcolor=blue!20]
\LaTeX ~submissions are mandatory. Submitting your assignment in another format will be graded no higher than R.
\end{mdframed}

\section{Lab Book S1}
In this Lab Book you will work with text segmentation, the first step of converting written text to speech signal.

\begin{problem}{1}{3}{Tokenizing a sentence}

\subsubsection*{Task:}
Using the definition of a ``word'' discussed in class (based on Taylor (2009)), tokenize and verbalize (without code) the following sentences into ``words'':
\begin{itemize}
    \item Space is big
    \item You just won't believe how vastly, hugely, mind-bogglingly big it is.
    \item I'm gonna write to test@example.com if you pay me \texteuro42.42.
    \item In 1903 Polish physicist Marie Curie became the 1st woman to win a Nobel Prize, she was 36y.o. then.
\end{itemize}
\end{problem}
\begin{solution}
    \begin{enumerate}
        \item Tokenizing
        \begin{itemize}
            \item \texttt{<SPACE> <IS> <BIG>}
            \item \texttt{<YOU> <JUST> <WON'T> <BELIEVE> <HOW> <VASTLY> <,> <HUGELY> <,> <MIND-BOGGLINGLY> \\<BIG> <IT> <IS> <.>}
            \item \texttt{<I'M> <GONNA> <WRITE> <TO> <TEST@EXAMPLE.COM> <IF> <YOU> <PAY> <ME> <\texteuro42.42> <.>}
            \item \texttt{<IN> <1903> <POLISH> <PHYSICIST> <MARIE> <CURIE> <BECAME> <THE> <1ST> <WOMAN> \\<TO> <WIN> <A> <NOBEL> <PRIZE> <,> <SHE> <WAS> <36Y.O.> <THEN> <.>}
        \end{itemize}
        \item Verbalizing
        \begin{itemize}
            \item \texttt{<SPACE> <IS> <BIG>}
            \item \texttt{<YOU> <JUST> <WON'T> <BELIEVE> <HOW> <VASTLY> <HUGELY> <MIND-BOGGLINGLY> <BIG> \\<IT> <IS>}
            \item \texttt{<I'M> <GONNA> <WRITE> <TO> <TEST> <WORD\_AT> <EXAMPLE> <WORD\_DOT> <COM> <IF> \\<YOU> <PAY> <ME> <FORTY-TWO> <EUROS> <FORTY-TWO> <CENTS>}
            \item \texttt{<IN> <NINTEEN> <OH> <THREE> <POLISH\_COUNTRY> <PHYSICIST> <MARIE> <CURIE> \\<BECAME> <THE> <FIRST> <WOMAN> <TO> <WIN> <A> <NOBEL> <PRIZE> <SHE> <WAS> \\<THIRTYSIX> <YEARS> <OLD> <THEN>}
        \end{itemize}
    \end{enumerate}

\end{solution}


\begin{problem}{1}{7}{Exploring text normalization}

\subsubsection*{Preparation:}
We will adopt an ad-hoc approach, and assume that the text that we are working with is mostly natural language. So we will use regular expressions trying to find a pattern. 
Please, write your own regular expression and use \textit{only the standard/built in} regex libraries (e.g. \texttt{re} in Python). Don't use existing advanced text processing libraries such as Spacy or Keras. 

\subsubsection*{Task:}
Write a script that extracts the following features from the text:
\begin{itemize}
    \item Phone numbers in international format, They begin with a \texttt{+} sign and contain up to 15 digits separated by spaces or dashes. For simplicity
assume that only the \texttt{-} (``minus sign'') will be used. Examples are: \texttt{+31503638004, +31 50 363 80 04, +31 50-363-80-04}. You can also choose to limit yourself to European countries: if you do, be transparent about it. 
    \item Web links, such as \texttt{https://www.rug.nl/info/contact}, \\
        \texttt{https://en.wikipedia.org/wiki/Regular\_expression\#Syntax}, \\
        \texttt{https://www.google.com/search?q=python\%20re}.
    \end{itemize}

As always, write good comments in the code explaining how it works.
Prepare a set of sentences to test your script and submit it together with your code to the GitHub repository.
Make sure the code works "as is" and doesn't require editing paths, providing commandline argument and so on.


\end{problem}

\begin{solution}
    For your convenience, I've included the test content consisting of several sentences in the codes so you may not need to edit the path or commandline argument and so on.
\end{solution}
\end{document}
