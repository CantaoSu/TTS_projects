%%%%%%%%%%%%%%%%%%%%%%%%%%%%%%%%%%%%%%%%%%%%%%%%%%%%%%%%%%%%%%%%%%%%%%%%%%%%%%%%%%%%
% Do not alter this block (unless you're familiar with LaTeX)
\documentclass{../labbook}

%%%%%%%%%%%%%%%%%%%%%%%%%%%%%%%%%%%%%%%%%%%%%
%Fill in the appropriate information below
\lhead{YOUR NAME HERE}
\rhead{Speech Synthesis I} 
\chead{\textbf{Lab Book S2. Due: \textbf{FRI 08.12.2023 23:59} CET}}
%%%%%%%%%%%%%%%%%%%%%%%%%%%%%%%%%%%%%%%%%%%%%

\begin{document}
\begin{mdframed}[backgroundcolor=blue!20]
\LaTeX ~submissions are mandatory (when applicable). Submitting your assignment in another format will be graded no higher than R.
\end{mdframed}

\section{Lab Book S1}
In this Lab Book you will work with prosody prediction form text. 

\begin{problem}{1}{10}{Verbalizing with code}

\subsubsection*{Task:}
Take the phone numbers you found in the first lab book and verbalize them with a code. 
For example, +31503638004 could become "PLUS THIRTY ONE FIVE OH THREE SIX THREE EIGHT OH OH FOUR".
Note that grouping of the digits may differ: for example, the country code should be distinguished from the rest of the number:
don't read a Macedonian (country code "+392") phone number +3921231231 as "PLUS THIRTY NINE TWENTY ONE...".

Think of a digit grouping that a TTS system could adopt when reading the phone numbers out loud (\href{https://youtu.be/2OX8znJHNo0?si=EG2d8ElbaailUEhU}{here} is an example of digit groupings in American English)
and introduce the phrase break or pause information by adding tokens <BR> for breaks or <PAU> for pauses where you feel necessary. 

For example, a sentence "John, please come in" could be verbalized with additional prosodic information as "JOHN <BR> PLEASE <PAU> COME IN"

Don't forget to comment in the code and explain how it works and please motivate your choice for the digit grouping. 
Submit the code to the GitHub repository. 
\end{problem}
\end{document}
